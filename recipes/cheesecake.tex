%------------------------------------------
% information doc
\Title{Torta di Rose}{Mamma}
\CookingTime{30}
\PrepTime{180}
\CookingTempe{180}
\TypeCooking{Tortiera}
\NbPerson{?}
\Image{0 0 500 400}{images/placeholder.png}
%------------------------------------------

\begin{ingredient}
\begin{tabularx}{\textwidth} { 
  >{\raggedright\arraybackslash}X 
  >{\raggedright\arraybackslash}X}
 300 mL & latte\\
 1 cubetto  & lievito di birra\\
 4 C & zucchero\\
 500 g & farina\\
 1 C & burro morbido\\
 1 c & sale\\
 qb & scorza limone o zucchero vanillinato\\
\end{tabularx}
\rule{\textwidth}{1pt}
\vspace{0.25cm}
\underline{Ripieno:}
\begin{tabularx}{0.8\textwidth} { 
  >{\raggedright\arraybackslash}X 
  >{\raggedright\arraybackslash}X}
  200 g & semi di papavero\\
  400 mL & latte \\
  5 C & zucchero\\
  1 & limone (scorza)\\
  1 cc & cannella\\
  1-2 & mele\\
 \end{tabularx}
 \rule{\textwidth}{1pt}
\vspace{0.25cm}
\underline{Glassa:}
\begin{tabularx}{0.8\textwidth} { 
  >{\raggedright\arraybackslash}X 
  >{\raggedright\arraybackslash}X}
  75 mL & panna\\
  2 C & zucchero\\
 \end{tabularx}

\end{ingredient} %no space with \begin{recipe}
\begin{recipe}
\substep[Ripieno]{Macinare il papavero finemente.}
\substep{Aggiungere il latte, lo zucchero, il limone, la cannella e cuocere a fuoco lento per 2 ore circa.}
\substep{Quando il composto avrà raggiunto una consistenza cremosa, aggiungere le mele grattuggiate e correggere il sapore a piacere (con limone e cannella).}
\step{Mentre il ripieno cuoce, preparate l'impasto mescolando il latte, il lievito e lo zucchero}
\step{Aggiungere la farina, il burro, il sale e gli aromi ed impastare fino ad omogeneizzare il composto.}
\step{Lasciare riposare finché l'impasto non lievita (aumenta di volume).}	
\step{Dividere l'impasto in 3 pezzi e stenderlo (?? spessore?), spalmare il ripieno ed arrotolare l'impasto in piccoli rotolini.}
\step{Tagliare il rotolino in sezioni di 3 cm circa (i pezzi finali del rotolo devono essere un poco più lunghi).}
\step{Foderare una tortiera con carta forno (diametro??, quante??) e disporre le sezioni di rotolo partendo dal centro con i pezzi finali del rotolo e lasciare lievitare in tortiera finché non viene riempita dall'impasto.}
\step{Infornare a 180°C per 30 minuti, una volta sfornata spennellare con la mistura di panna e zucchero a velo finché ancora calda.}
\step{Lasciar stemperare e servire tiepida o a temperatura ambiente.}

\end{recipe}

\begin{notes}
\begin{itemize}
    \item Se il ripieno risulta troppo asciutto aggiungere latte, se troppo liquido lasciare cuocere.
    \item La torta va servita tiepido-fredda poiché calda si sfalda facilmente.
    \item \textbf{Variante Tradizionale:} Al posto del ripieno al papavero utilizzare un misto di burro e zucchero.
    \item \textbf{Variante alle Noci:} Al posto del ripieno al papavero utilizzare la crema alle noci (vedi ricettario bimby).
\end{itemize}
\end{notes}	